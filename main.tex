\documentclass[
	% -- opções da classe memoir --
	12pt,				% tamanho da fonte
	openright,			% capítulos começam em pág ímpar (página vazia se preciso)
% 	twoside,			% para impressão em recto e verso.
    oneside,
	a4paper,			% tamanho do papel. 
	% -- opções da classe abntex2 --
	chapter=TITLE,
% 	section=Title,
% 	subsection=Title,
% 	subsubsection=Title,
	% -- opções do pacote babel --
	english,			% idioma adicional para hifenização
% 	french,				% idioma adicional para hifenização
% 	spanish,			% idioma adicional para hifenização
	brazil				% o último idioma é o principal do documento
	]{./abntex2}

% ---
% Pacotes básicos 
% ---
\usepackage{lmodern}		% Usa a fonte Latin Modern			
\usepackage[T1]{fontenc}	% Selecao de codigos de fonte.
\usepackage[utf8]{inputenc}	% Codificacao do documento 
\usepackage{indentfirst}	% Indenta o primeiro parágrafo de cada seção.
\usepackage{color}			% Controle das cores
\usepackage{graphicx}		% Inclusão de gráficos
\usepackage{microtype} 		% para melhorias de justificação
\usepackage{setspace}       % para espacamento simples no resumo
\usepackage{pdfpages}
% ---
		
% ---
% Pacotes adicionais, usados apenas no exemplo
% ---
\usepackage{lipsum} % para geração de dummy text
% ---

% ---
% Pacotes de citações
% ---
%\usepackage[brazilian,hyperpageref]{backref} % Mostra a página onde cada citação foi feita
\usepackage[num,abnt-etal-list=0,bibjustif]{./abntex2cite} % Citações numéricas (ordem de apresentação)
%\usepackage[alf,abnt-etal-list=0,bibjustif]{./abntex2cite} % Citações autor-data (ordem alfabética)

% --- 
% CONFIGURAÇÕES DE PACOTES
% --- 

% % ---
% % Configurações do pacote backref, SE FOR USAR DESCOMENTE TODO ESSE TRECHO
% % Usado sem a opção hyperpageref de backref
% \renewcommand{\backrefpagesname}{Citado na(s) página(s):~}
% % Texto padrão antes do número das páginas
% \renewcommand{\backref}{}
% % Define os textos da citação
% \renewcommand*{\backrefalt}[4]{
% 	\ifcase #1 %
% 		Nenhuma citação no texto.%
% 	\or
% 		Citado na página #2.%
% 	\else
% 		Citado #1 vezes nas páginas #2.%
% 	\fi}%
% % ---


% ---
% Informações de dados para CAPA e FOLHA DE ROSTO
% ---
\usepackage{abntex2ime}
% ---



% ---
% EXEMPLO PARA CURSO DE POS GRADUAÇÃO
% ---

% \instituicao{Instituto Militar de Engenharia}
% \programapgdepartamento{Engenharia de Transportes}
% \nivelestudo{Doutorado} % Graduação, Mestrado ou Doutorado
% % ---
% \titulo{Modelo Canônico de Trabalho Acadêmico com \abnTeX\space \versaoDocumento}
% \palavraschave{arp,sarp,iot,vant,tarefas cooperativas,agentes inteligentes}
% \keywords{unmanned systems,unmanned vehicles,uav,uas,cooperative tasks,intelligent agents}
% % ---
% \autores{Fulano de}{Silva}% 1+ autores
% % \autor{Fulano de}{Tal} %{nome}{sobrenome}
% \orientadores{Sicrano,Beltrano}{Santos,Oliveira}{Ph.D.,D.Sc.}%{nomes}{sobrenomes}{títulos}
% % ---
% \local{Rio de Janeiro}
% \data{2020}
% \datadefesa{30 de fevereiro de 2020}
% \bancadeexaminadores{
%     Prof. \textbf{Orientador 1} - D.Sc. do IME - Presidente,
%     Prof. \textbf{Orientador 2} - D.Sc. do LNCC,
%     Prof. \textbf{Professor 1} - Ph.D. do IMPA,
%     Prof. \textbf{Professor 2} - D.Sc. do LNCC,
%     Prof. \textbf{Professor 3} - D.Sc. do IME,
%     Prof. \textbf{Professor 4} - D.Sc. da PUC
% }

% ---



% ---
% EXEMPLO PARA PROJETO DE FIM DE CURSO
% ---

\instituicao{Instituto Militar de Engenharia}
\programapgdepartamento{Engenharia de Fortificação e Construção}
\nivelestudo{Graduação} % Graduação, Mestrado ou Doutorado
% ---
\titulo{Modelo Canônico de Trabalho Acadêmico com \abnTeX\space \versaoDocumento}
\palavraschave{arp,sarp,iot,vant,tarefas cooperativas,agentes inteligentes}
\keywords{unmanned systems,unmanned vehicles,uav,uas,cooperative tasks,intelligent agents}
% ---
\autores{Fulano da,Ciclano}{Silva,Pereira}% {nome}{sobrenome} 1+
\orientadores{Sicrano,Beltrano}{Santos,Oliveira}{Ph.D.,D.Sc.}%{nomes}{sobrenomes}{títulos} 1+
% ---
\local{Rio de Janeiro}
\data{2020}
\datadefesa{30 de fevereiro de 2020}
\bancadeexaminadores{
    Prof. \textbf{Orientador 1} - D.Sc. do IME - Presidente,
    Prof. \textbf{Orientador 2} - D.Sc. do LNCC,
    Prof. \textbf{Professor 1} - Ph.D. do IMPA,
    Prof. \textbf{Professor 2} - D.Sc. do LNCC,
    Prof. \textbf{Professor 3} - D.Sc. do IME,
    Prof. \textbf{Professor 4} - D.Sc. da PUC
}

% ---



% ---
% Configurações de aparência do PDF final

% % alterando o aspecto da cor azul
% \definecolor{blue}{RGB}{41,5,195}

% % informações do PDF
\makeatletter
\hypersetup{
    %pagebackref=true,
    % pdftitle={\@title}, 
    % pdfauthor={\@author},
    % pdfsubject={\imprimirpreambulo},
    % pdfcreator={LaTeX with abnTeX2},
    colorlinks=true, % false: boxed links; true: colored links
    linkcolor=black, % color of internal links
    citecolor=black,	% color of links to bibliography
    filecolor=black, % color of file links
    urlcolor=black,
    bookmarksdepth=4
}
\makeatother
% % --- 
% % --- 

% ---
% Posiciona figuras e tabelas no topo da página quando adicionadas sozinhas
% em um página em branco. Ver https://github.com/abntex/abntex2/issues/170
\makeatletter
\setlength{\@fptop}{5pt} % Set distance from top of page to first float
\makeatother
% ---

% ---
% Possibilita criação de Quadros e Lista de quadros.
% Ver https://github.com/abntex/abntex2/issues/176
%
\newcommand{\quadroname}{Quadro}
\newcommand{\listofquadrosname}{Lista de quadros}

\newfloat[chapter]{quadro}{loq}{\quadroname}
\newlistof{listofquadros}{loq}{\listofquadrosname}
\newlistentry{quadro}{loq}{0}

% configurações para atender às regras da ABNT
\setfloatadjustment{quadro}{\centering}
\counterwithout{quadro}{chapter}
\renewcommand{\cftquadroname}{\quadroname\space} 
\renewcommand*{\cftquadroaftersnum}{\hfill--\hfill}

\setfloatlocations{quadro}{hbtp} % Ver https://github.com/abntex/abntex2/issues/176
% ---

% --- 
% Espaçamentos entre linhas e parágrafos 
% --- 

% O tamanho do parágrafo é dado por:
\setlength{\parindent}{1.3cm}

% Controle do espaçamento entre um parágrafo e outro:
\setlength{\parskip}{0.2cm}  % tente também \onelineskip

% ---
% compila o indice
% ---
\makeindex
% ---

% ----
% Início do documento
% ----
\begin{document}

% Seleciona o idioma do documento (conforme pacotes do babel)
%\selectlanguage{english}
\selectlanguage{brazil}

% Retira espaço extra obsoleto entre as frases.
\frenchspacing 

% ----------------------------------------------------------
% ELEMENTOS PRÉ-TEXTUAIS
% ----------------------------------------------------------
% \pretextual

% ---
% Capa
% ---
\imprimircapa
% ---

% ---
% Folha de rosto
% (o * indica que haverá a ficha bibliográfica)
% ---
\imprimirfolhaderosto*
% ---

% ---
% Inserir a ficha bibliografica
% ---

% \begin{fichacatalografica}
%     \includepdf{fig_ficha_catalografica.pdf}
% \end{fichacatalografica}

\imprimirfichacatalografica
% ---

% ---
% Inserir folha de aprovação
% ---

% \begin{folhadeaprovacao}
% \includepdf{folhadeaprovacao_final.pdf}
% \end{folhadeaprovacao}
%
\imprimirfolhadeaprovacao
% ---

% ---
% Dedicatória
% ---
\begin{dedicatoria}
   \vspace*{\fill}
   \centering
   \noindent
   \textit{ Este trabalho é dedicado às crianças adultas que,\\
   quando pequenas, sonharam em se tornar cientistas.} \vspace*{\fill}
\end{dedicatoria}
% ---

% ---
% Agradecimentos
% ---
\begin{agradecimentos}
Os agradecimentos principais são direcionados à Gerald Weber, Miguel Frasson, Leslie H. Watter, Bruno Parente Lima, Flávio de Vasconcellos Corrêa, Otavio Real Salvador, Renato Machnievscz\footnote{Os nomes dos integrantes do primeiro projeto abn\TeX\ foram extraídos de \url{http://codigolivre.org.br/projects/abntex/}} e todos aqueles que contribuíram para que a produção de trabalhos acadêmicos conforme as normas ABNT com \LaTeX\ fosse possível.

Agradecimentos especiais são direcionados ao Centro de Pesquisa em Arquitetura da Informação\footnote{\url{http://www.cpai.unb.br/}} da Universidade de Brasília (CPAI), ao grupo de usuários \emph{latex-br}\footnote{\url{http://groups.google.com/group/latex-br}} e aos novos voluntários do grupo \emph{\abnTeX}\footnote{\url{http://groups.google.com/group/abntex2} e \url{http://www.abntex.net.br/}}~que contribuíram e que ainda contribuirão para a evolução do \abnTeX.

\end{agradecimentos}
% ---

% ---
% Epígrafe
% ---
\begin{epigrafe}
    \vspace*{\fill}
	\begin{flushright}
		\textit{``Não vos amoldeis às estruturas deste mundo, \\
		mas transformai-vos pela renovação da mente, \\
		a fim de distinguir qual é a vontade de Deus: \\
		o que é bom, o que Lhe é agradável, o que é perfeito.\\
		(Bíblia Sagrada, Romanos 12, 2)}
	\end{flushright}
\end{epigrafe}
% ---

% ---
% RESUMOS
% ---

% resumo em português
\setlength{\absparsep}{18pt} % ajusta o espaçamento dos parágrafos do resumo
\begin{resumo}
\SingleSpacing
 Segundo a \citeonline[3.1-3.2]{NBR6028:2003}, o resumo deve ressaltar o
 objetivo, o método, os resultados e as conclusões do documento. A ordem e a extensão
 destes itens dependem do tipo de resumo (informativo ou indicativo) e do
 tratamento que cada item recebe no documento original. O resumo deve ser
 precedido da referência do documento, com exceção do resumo inserido no
 próprio documento. (\ldots) As palavras-chave devem figurar logo abaixo do
 resumo, antecedidas da expressão Palavras-chave:, separadas entre si por
 ponto e finalizadas também por ponto.

 \textbf{Palavras-chave}: \imprimirpalavraschave
\end{resumo}

% resumo em inglês
\begin{resumo}[Abstract]
 \begin{otherlanguage*}{english}
%  \linespread{1.3}
\SingleSpacing
 Segundo a \citeonline[3.1-3.2]{NBR6028:2003}, o resumo deve ressaltar o
 objetivo, o método, os resultados e as conclusões do documento. A ordem e a extensão
 destes itens dependem do tipo de resumo (informativo ou indicativo) e do
 tratamento que cada item recebe no documento original. O resumo deve ser
 precedido da referência do documento, com exceção do resumo inserido no
 próprio documento. (\ldots) As palavras-chave devem figurar logo abaixo do
 resumo, antecedidas da expressão Palavras-chave:, separadas entre si por
 ponto e finalizadas também por ponto.
   This is the english abstract.
   \vspace{\onelineskip}
 
   \noindent 
   \textbf{Keywords}: \imprimirkeywords
 \end{otherlanguage*}
\end{resumo}

% ---
% inserir lista de ilustrações
% ---
\pdfbookmark[0]{\listfigurename}{lof}
\listoffigures*
\cleardoublepage
% ---

% ---
% inserir lista de quadros
% ---
\pdfbookmark[0]{\listofquadrosname}{loq}
\listofquadros*
\cleardoublepage
% ---

% ---
% inserir lista de tabelas
% ---
\pdfbookmark[0]{\listtablename}{lot}
\listoftables*
\cleardoublepage
% ---

% ---
% inserir lista de abreviaturas e siglas
% ---
\begin{siglas}
  \item[ABNT] Associação Brasileira de Normas Técnicas
  \item[abnTeX] ABsurdas Normas para TeX
\end{siglas}
% ---

% ---
% inserir lista de símbolos
% ---
\begin{simbolos}
  \item[$ \Gamma $] Letra grega Gama
  \item[$ \Lambda $] Lambda
  \item[$ \zeta $] Letra grega minúscula zeta
  \item[$ \in $] Pertence
\end{simbolos}
% ---


% ---
% inserir o sumario
% ---

\makeatletter
\let\oldcontentsline\contentsline
\def\contentsline#1#2{%
    \oldcontentsline{#1}{\MakeTextUppercase{#2}}%
}
\makeatother

\pdfbookmark[0]{\contentsname}{toc}
\tableofcontents*
\cleardoublepage
% ---

% ----------------------------------------------------------
% ELEMENTOS TEXTUAIS
% ----------------------------------------------------------
\textual

% ----------------------------------------------------------
% Introdução (exemplo de capítulo sem numeração, mas presente no Sumário)
% ----------------------------------------------------------
\chapter{Introdução}
% ----------------------------------------------------------

Este documento e seu código-fonte são exemplos de referência de uso da classe
\textsf{abntex2} e do pacote \textsf{abntex2cite}. O documento 
exemplifica a elaboração de trabalho acadêmico (tese, dissertação e outros do
gênero) produzido conforme a ABNT NBR 14724:2011 \emph{Informação e documentação
- Trabalhos acadêmicos - Apresentação}.

A expressão ``Modelo Canônico'' é utilizada para indicar que \abnTeX\ não é
modelo específico de nenhuma universidade ou instituição, mas que implementa tão
somente os requisitos das normas da ABNT. Uma lista completa das normas
observadas pelo \abnTeX\ é apresentada em \citeonline{abntex2classe}.

Sinta-se convidado a participar do projeto \abnTeX! Acesse o site do projeto em
\url{http://www.abntex.net.br/}. Também fique livre para conhecer,
estudar, alterar e redistribuir o trabalho do \abnTeX, desde que os arquivos
modificados tenham seus nomes alterados e que os créditos sejam dados aos
autores originais, nos termos da ``The \LaTeX\ Project Public
License''\footnote{\url{http://www.latex-project.org/lppl.txt}}.

Encorajamos que sejam realizadas customizações específicas deste exemplo para
universidades e outras instituições --- como capas, folha de aprovação, etc.
Porém, recomendamos que ao invés de se alterar diretamente os arquivos do
\abnTeX, distribua-se arquivos com as respectivas customizações.
Isso permite que futuras versões do \abnTeX~não se tornem automaticamente
incompatíveis com as customizações promovidas. Consulte
\citeonline{abntex2-wiki-como-customizar} para mais informações.

Este documento deve ser utilizado como complemento dos manuais do \abnTeX\ 
\cite{abntex2classe,abntex2cite,abntex2cite-alf} e da classe \textsf{memoir}
\cite{memoir}. 

Esperamos, sinceramente, que o \abnTeX\ aprimore a qualidade do trabalho que
você produzirá, de modo que o principal esforço seja concentrado no principal:
na contribuição científica.

Equipe \abnTeX 

Lauro César Araujo
\chapter{Capítulo Exemplo}
\noindent
Lorem ipsum dolor sit amet, consectetuer adipiscing elit. Ut purus elit, vestibulum ut, placerat ac, adipiscing vitae, felis \cite{Salles2014}. Curabitur dictum gravida mauris. Nam arcu libero, nonummy eget, consectetuer id, vulputate a, magna. Donec vehicula augue eu neque. Pellentesque habitant morbi tristique senectus et netus et malesuada fames ac turpis egestas. Mauris ut leo. Cras viverra metus rhoncus sem. Nulla et lectus vestibulum urna fringilla ultrices. Phasellus eu tellus sit amet tortor gravida placerat. Integer sapien est, iaculis in, pretium quis, viverra ac, nunc.

Praesent eget sem vel leo ultrices bibendum. Aenean faucibus. Morbi dolor nulla, malesuada eu, pulvinar at, mollis ac, nulla. Curabitur auctor semper nulla. Donec varius orci eget risus. Duis nibh mi, congue eu, accumsan eleifend, sagittis quis, diam \cite{Justel2014}. Duis eget orci sit amet orci dignissim rutrum. Nam dui ligula, fringilla a, euismod sodales, sollicitudin vel, wisi. Morbi auctor lorem non justo. Nam lacus libero, pretium at, lobortis vitae, ultricies et, tellus. Donec aliquet, tortor sed accumsan bibendum, erat ligula aliquet magna, vitae ornare odio metus a mi. Morbi ac orci et nisl hendrerit mollis. Suspendisse ut massa.

\section{Seção Exemplo}
Segundo \citeauthor{Goldschmidt2005}, cras nec ante. Pellentesque a nulla. Cum sociis natoque penatibus et magnis dis parturient montes, nascetur ridiculus mus. Aliquam tincidunt urna \cite{Rakocevic2014}. Nulla ullamcorper vestibulum turpis. Pellentesque cursus luctus mauris. Nulla malesuada porttitor diam. Donec felis erat, congue non, volutpat at, tincidunt tristique, libero. Vivamus viverra fermentum felis. Donec nonummy pellentesque ante. Phasellus adipiscing semper elit. Proin fermentum massa ac quam \cite{Lara2014}. Sed diam turpis, molestie vitae, placerat a, molestie nec, leo. Maecenas lacinia. Nam ipsum ligula, eleifend at, accumsan nec, suscipit a, ipsum.

Morbi blandit ligula feugiat magna. De acordo com  \citeauthor{Soares2013} nunc eleifend consequat lorem. Sed lacinia nulla vitae enim. Pellentesque tincidunt purus vel magna. Integer non enim. Praesent euismod nunc eu purus. Donec bibendum quam in tellus \cite{Yoko2003}. Nullam cursus pulvinar lectus. Donec et mi. Nam vulputate metus eu enim. Vestibulum pellentesque felis eu massa. Quisque ullamcorper placerat ipsum. Cras nibh. Morbi vel justo vitae lacus tincidunt ultrices. Lorem ipsum dolor sit amet, consectetuer adipiscing elit. In hac habitasse platea dictumst. Em \citeauthor{Dias2013} integer tempus convallis augue. Etiam facilisis. Nunc elementum fermentum wisi. Aenean placerat.

\begin{figure}[!ht]
	\centering
	\includegraphics[width=0.4\textwidth]{img/artigo1.eps}   
	\caption[Rótulo da Figura, descrevendo a figura]{Rótulo da Figura, descrevendo a figura \cite{icse2015}.}
	\label{fig:figura1}
\end{figure}

Ut imperdiet, enim sed gravida sollicitudin, felis odio placerat quam, ac pulvinar elit purus eget enim \cite{Gubitoso1992}. Nunc vitae tortor. Proin tempus nibh sit amet nisl, como se pode ver na Figura \ref{fig:figura1}. Vivamus quis tortor vitae risus porta vehicula. Fusce mauris. Vestibulum luctus nibh at lectus. Sed bibendum, nulla a faucibus semper, leo velit ultricies tellus, ac venenatis arcu wisi vel nisl \cite{icse2015}. Aliquam pellentesque, augue quis sagittis posuere, turpis lacus congue quam, in hendrerit risus eros eget felis. Maecenas eget erat in sapien mattis porttitor\footnotemark. Vestibulum porttitor. Nulla facilisi. Sed a turpis eu lacus commodo facilisis. Morbi fringilla, wisi in dignissim interdum, justo lectus sagittis dui, et vehicula libero dui cursus dui.

\footnotetext{Pellentesque a nulla cum sociis natoque penatibus et magnis dis parturient, nascetur ridiculus mus.}

Mauris tempor ligula sed lacus. Duis cursus enim ut augue. Cras ac magna. Cras nulla. Nulla egestas. Curabitur a leo. Quisque egestas wisi eget nunc. Nam feugiat lacus vel est. Curabitur consectetuer \cite{Araujo2015}. Gini eect on Secondary school enrollment Lorem ipsum dolor sit amet, consectetuer adipiscing elit. Ut purus elit, vestibulum ut, placerat ac, adipiscing vitae, felis. Curabitur dictum gravida mauris. Nam arcu libero, nonummy eget, consectetuer id, vulputate a, magna \cite{Folha2015}. Donec vehicula augue eu neque.

Pellentesque habitant morbi tristique senectus et netus et malesuada fames ac turpis egestas. Mauris ut leo. Integer sapien est, iaculis in, pretium quis, viverra ac, nunc. Praesent eget sem vel leo ultrices bibendum. Aenean faucibus. Morbi dolor nulla, malesuada eu, pulvinar at, mollis ac, nulla. Sed ut perspiciatis unde omnis iste natus error sit voluptatem accusantium doloremque laudantium, totam rem aperiam, eaque ipsa quae ab illo inventore veritatis et quasi architecto beatae vitae dicta sunt explicabo.

\subsection{Subseção Exemplo}
Cras viverra metus rhoncus sem. Nulla et lectus vestibulum urna fringilla ultrices. Phasellus eu tellus sit amet tortor gravida placerat (como mostrado na Tabela \ref{tab:tabela1}). Curabitur auctor semper nulla. Donec varius orci eget risus. Duis nibh mi, congue eu, accumsan eleifend, sagittis quis, diam. Duis eget orci sit amet orci dignissim rutrum. Nam dui ligula, fringilla a, euismod sodales, sollicitudin vel, wisi. Morbi auctor lorem non justo. Nam lacus libero, pretium at, lobortis vitae, ultricies et, tellus. Donec aliquet, tortor sed accumsan bibendum, erat ligula aliquet magna, vitae ornare odio metus a mi.

\begin{table}[ht]
\centering
\caption{Título da Tabela, descrevendo a tabela.}
\vspace{0.5cm}
\begin{tabular}{r|lr}
 
Posição & País & IDH \\
\hline
1 & Noruega        & .955 \\
2 & Austrália  & .938 \\
3 & EUA            & .937 \\
4 & Holanda        & .921 \\
5 & Alemanha       & .920 
 
\end{tabular}
\label{tab:tabela1}
\end{table}

Nemo enim ipsam voluptatem quia voluptas sit aspernatur aut odit aut fugit, sed quia consequuntur magni dolores eos qui ratione voluptatem sequi nesciunt. Neque porro quisquam est, qui dolorem ipsum quia dolor sit amet, consectetur, adipisci velit, sed quia non numquam eius modi tempora incidunt ut labore et dolore magnam aliquam quaerat voluptatem. Ut enim ad minima veniam, quis nostrum exercitationem ullam corporis suscipit laboriosam, nisi ut aliquid ex ea commodi consequatur? Quis autem vel eum iure reprehenderit qui in ea voluptate velit esse quam nihil molestiae consequatur, vel illum qui dolorem eum fugiat quo voluptas nulla pariatur?

At vero eos et accusamus et iusto odio dignissimos ducimus qui blanditiis praesentium voluptatum deleniti atque corrupti quos dolores et quas molestias excepturi sint occaecati cupiditate non provident, similique sunt in culpa qui officia deserunt mollitia animi, id est laborum et dolorum fuga. Et harum quidem rerum facilis est et expedita distinctio. Nam libero tempore, cum soluta nobis est eligendi optio cumque nihil impedit quo minus id quod maxime placeat facere possimus, omnis voluptas assumenda est, omnis dolor repellendus. Temporibus autem quibusdam et aut officiis debitis aut rerum necessitatibus saepe eveniet ut et voluptates repudiandae sint et molestiae non recusandae.

\subsubsection{Subsubseção Exemplo}
\subsubsubsection{Subsubsubseção Exemplo}
\chapter{Conteúdos específicos do modelo de trabalho acadêmico}\label{cap_trabalho_academico}

\section{Quadros}

Este modelo vem com o ambiente \texttt{quadro} e impressão de Lista de quadros configurados por padrão. Verifique um exemplo de utilização:

\begin{quadro}[htb]
\caption{\label{quadro_exemplo}Exemplo de quadro}
\begin{tabular}{|c|c|c|c|}
	\hline
	\textbf{Pessoa} & \textbf{Idade} & \textbf{Peso} & \textbf{Altura} \\ \hline
	Marcos & 26    & 68   & 178    \\ \hline
	Ivone  & 22    & 57   & 162    \\ \hline
	...    & ...   & ...  & ...    \\ \hline
	Sueli  & 40    & 65   & 153    \\ \hline
\end{tabular}
\fonte{Autor.}
\end{quadro}

Este parágrafo apresenta como referenciar o quadro no texto, requisito
obrigatório da ABNT. 
Primeira opção, utilizando \texttt{autoref}: Ver o \autoref{quadro_exemplo}. 
Segunda opção, utilizando  \texttt{ref}: Ver o Quadro \ref{quadro_exemplo}.

% ---
% Capitulo de revisão de literatura
% ---
\chapter{Lorem ipsum dolor sit amet}
% ---

% ---
\section{Aliquam vestibulum fringilla lorem}
% ---

\lipsum[1]

\lipsum[2-3]
% ---
% Conclusão
% ---
\chapter{Conclusão}
% ---

\lipsum[31-33]

% ----------------------------------------------------------
% ELEMENTOS PÓS-TEXTUAIS
% ----------------------------------------------------------
\postextual
% ----------------------------------------------------------

% ----------------------------------------------------------
% Referências bibliográficas
% ----------------------------------------------------------
\bibliography{refs}

% ----------------------------------------------------------
% Apêndices
% ----------------------------------------------------------
\begin{apendicesenv}
    \partapendices
    \chapter{Apêndice Exemplo}

Curabitur tortor. Pellentesque nibh. Aenean quam. In scelerisque sem at dolor. Maecenas mattis. Sed convallis tristique sem. Proin ut ligula vel nunc egestas porttitor. Morbi lectus risus, iaculis vel, suscipit quis, luctus non, massa. Fusce ac turpis quis ligula lacinia aliquet. Mauris ipsum. Nulla metus metus, ullamcorper vel, tincidunt sed, euismod in, nibh. Quisque volutpat condimentum velit.

Class aptent taciti sociosqu ad litora torquent per conubia nostra, per inceptos himenaeos. Nam nec ante. Sed lacinia, urna non tincidunt mattis, tortor neque adipiscing diam, a cursus ipsum ante quis turpis. Nulla facilisi. Ut fringilla. Suspendisse potenti. Nunc feugiat mi a tellus consequat imperdiet. Vestibulum sapien. Proin quam. Etiam ultrices. Suspendisse in justo eu magna luctus suscipit. Sed lectus. Integer euismod lacus luctus magna.

Lorem ipsum dolor sit amet, consectetur adipiscing elit. Integer nec odio. Praesent libero. Sed cursus ante dapibus diam. Sed nisi. Nulla quis sem at nibh elementum imperdiet. Duis sagittis ipsum. Praesent mauris. Fusce nec tellus sed augue semper porta. Mauris massa. Vestibulum lacinia arcu eget nulla. Class aptent taciti sociosqu ad litora torquent per conubia nostra, per inceptos himenaeos. Curabitur sodales ligula in libero. Sed dignissim lacinia nunc.

\chapter{Apêndice Exemplo 02}

Curabitur tortor. Pellentesque nibh. Aenean quam. In scelerisque sem at dolor. Maecenas mattis. Sed convallis tristique sem. Proin ut ligula vel nunc egestas porttitor. Morbi lectus risus, iaculis vel, suscipit quis, luctus non, massa. Fusce ac turpis quis ligula lacinia aliquet. Mauris ipsum. Nulla metus metus, ullamcorper vel, tincidunt sed, euismod in, nibh. Quisque volutpat condimentum velit.

Class aptent taciti sociosqu ad litora torquent per conubia nostra, per inceptos himenaeos. Nam nec ante. Sed lacinia, urna non tincidunt mattis, tortor neque adipiscing diam, a cursus ipsum ante quis turpis. Nulla facilisi. Ut fringilla. Suspendisse potenti. Nunc feugiat mi a tellus consequat imperdiet. Vestibulum sapien. Proin quam. Etiam ultrices. Suspendisse in justo eu magna luctus suscipit. Sed lectus. Integer euismod lacus luctus magna.

Lorem ipsum dolor sit amet, consectetur adipiscing elit. Integer nec odio. Praesent libero. Sed cursus ante dapibus diam. Sed nisi. Nulla quis sem at nibh elementum imperdiet. Duis sagittis ipsum. Praesent mauris. Fusce nec tellus sed augue semper porta. Mauris massa. Vestibulum lacinia arcu eget nulla. Class aptent taciti sociosqu ad litora torquent per conubia nostra, per inceptos himenaeos. Curabitur sodales ligula in libero. Sed dignissim lacinia nunc.


\chapter{Quisque libero justo}

\lipsum[50]


\chapter{Nullam elementum urna vel imperdiet sodales elit ipsum pharetra ligula ac pretium ante justo a nulla curabitur tristique arcu eu metus}

\lipsum[54-56]
\end{apendicesenv}
% ---

% ----------------------------------------------------------
% Anexos
% ----------------------------------------------------------
\begin{anexosenv}
    \partanexos
    \chapter{Anexo Exemplo}

\lipsum[3-9]

\chapter{Anexo Exemplo 02}

\lipsum[1-4]


\chapter{Morbi ultrices rutrum lorem.}

\lipsum[29-30]


\chapter{Cras non urna sed feugiat cum sociis natoque penatibus et magnis dis parturient montes nascetur ridiculus mus}

\lipsum[31-32]

\chapter{Fusce facilisis lacinia dui}

\lipsum[32-36]
\end{anexosenv}

%---------------------------------------------------------------------
% INDICE REMISSIVO
%---------------------------------------------------------------------
\phantompart
\printindex
%---------------------------------------------------------------------

\end{document}
